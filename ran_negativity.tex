\documentclass[twocolumn,superscriptaddress,prb,10pt]{revtex4-1}
%\usepackage{verbatim}
\usepackage{amsmath,amssymb}
%\usepackage{unicode-math}
\usepackage{graphicx}
\usepackage{color}
\usepackage[colorlinks,bookmarks=false,citecolor=blue,linkcolor=red,urlcolor=blue]{hyperref}
\usepackage{times}





%\usepackage[dvips]{graphics}



%%%%%%%%%%%%%   useful shortcuts %%%%%%%%%%%%%%%%%%%%%%%%%%%%%%%%%

\def \h{\hbar}   %  \h won't be used for any greek letter
\def \refe{\eqref}
\def \trm{\textrm}
\def \f{\frac}
\def \hf{\tfrac{1}{2}}    \def \HF{\dfrac{1}{2}}
\def \u{\uparrow}
\def \d{\downarrow}

\def \ord{\mathcal{O}}
\newcommand{\ra}{\rightarrow}   \newcommand{\lra}{\longrightarrow}  

\def\lba{\left(}    \def\rba{\right)}
\def\lbc{\left[}    \def\rbc{\right]}
\def\lbb{\left\{}    \def\rbb{\right\}}

\def\tr{\textrm{Tr}}
\def\refe{\eqref}

\newcommand{\bra}[1]{\langle\left.{#1}\right|}
\newcommand{\ket}[1]{\left|{#1}\right.\rangle}
\newcommand{\xpct}[1]{\langle{#1}\rangle}    % expectatn value

%\DeclareMathOperator{\tr}{tr}

%%%%%%%%%%%%%%%%%%%%%%%%%%%%%%%%%%%%%%%%%%%%%%%%%%%%%%%

\newcommand{\vp}{{\bf p}}  % usual vector quantities
\newcommand{\vq}{{\bf q}}  % double bracketing not required with \vec
\newcommand{\vk}{{\bf k}}  % but required with \bf

\renewcommand{\vr}{{\bf r}} 
\newcommand{\vx}{{\bf x}}

\newcommand{\hc}{\hat{c}}  \newcommand{\hcd}{\hat{c}^\dag} 
\newcommand{\hd}{\hat{d}}  \newcommand{\hdd}{\hat{d}^\dag} 






\begin{document}

\title{The logarithmic negativity in random spin chains} 

%\author{Vincenzo Alba}
%\affiliation{International School for Advanced Studies (SISSA),
%Via Bonomea 265, 34136, Trieste, Italy, 
%INFN, Sezione di Trieste}

\author{Authors}

\date{\today}




\begin{abstract} 


\end{abstract}

% \pacs{73.43.Cd, 71.10.Pm  {\tt check!}}

\maketitle


%########################################################################
\section{Introduction}


%########################################################################
\section{The disordered $XX$ chain}
\label{dis-XX}

The disordered $XX$ chain is defined by the Hamiltonian 
%
\begin{equation}
{\mathcal H}_{XX}=\sum\limits_{i=1}^{L-1}J_i(S^x_iS^x_{i+1}+S^y_i
S^y_{i+1})+h\sum\limits_{i=1}^{L}S_i^z, 
\label{xx_ham}
\end{equation}
%
with $S^{x,y,z}_i\equiv\sigma_i^{x,y,z}/2$, $\sigma_i^\alpha$ 
being the Pauli matrices acting on site $i$. For periodic 
boundary conditions one has an extra term in Eq.~\eqref{xx_ham} 
connecting site $L$ with site $1$. Hereafter we fix $h=0$ and 
choose $J_i$ uniformily distributed in $[0,1]$. After the Jordan-Wigner 
transformation 
%
\begin{equation}
c_i=\Big(\prod\limits_{m=1}^{i-1}\sigma^z_m\Big)
\frac{\sigma_i^x-i\sigma_i^y}{2},
\label{j-wigner}
\end{equation}
%
\eqref{xx_ham} is recast in the free-fermionic form 
%
\begin{equation}
{\mathcal H}_{XX}=\frac{1}{2}\sum\limits_{i=1}^{L-1}J_i(c^\dagger_i 
c_{i+1}+c^\dagger_{i+1}c_i)+\frac{h}{2}\sum\limits_{i=1}^{L-1}
c^\dagger_i c_i,
\label{xx_fer}
\end{equation}
%
with $c_i$ spinless fermionic operators satisfying the canonical 
anticommutation relations $\{c_m,c^\dagger_n\}=\delta_{m,n}$. 
The mapping between Eq.~\eqref{xx_ham} and Eq.~\eqref{xx_fer} 
is exact apart from boundary terms (that we neglect here) giving a 
vanishing contribution (as $1/L$) to physical quantities in the  large 
chain limit. 

By imposing that the single-particle eigenstates $|\Psi_q\rangle$ (with $q$ 
an integer labelling the different eigenstates) of~\eqref{xx_fer} are of 
the form 
%
\begin{equation}
\eta_q^\dagger|0\rangle\equiv|\Psi_q\rangle=\sum_i\Phi_q(i)c_i^\dagger|0\rangle,
\end{equation}
%
with $|0\rangle$ the vacuum and $\Phi_q(i)$ the eigenstate amplitudes. 
The Schr\"odinger equation gives the equation for $\Phi_q(i)$ as 
%
\begin{equation}
\label{xx-eig}
(J_i\Phi_q(i+1)+J_{i-1}\Phi_q(i-1))/2=\epsilon_q\Phi_q(i),\quad i=1,\dots,L, 
\end{equation}
%
and $J_L=0$. Eq.~\eqref{xx-eig} corresponds to finding the eigenvectors 
of the banded matrix $T=(J_j\delta_{i,j+1}+J_{j-1}\delta_{i,j-1})/2$. The eigenvalues 
corresponds to the single-particle eigenergies $\epsilon_q$. 

The ground state $|GS\rangle$ of~\eqref{xx_ham} is obtained by filling all the 
negative modes $\epsilon_q$ as 
%
\begin{equation}
|GS\rangle=\eta^\dagger_{q_M}\eta^\dagger_{q_{M-1}}\cdots\eta_{q_1}^\dagger|0\rangle.
\end{equation}
%
One has the anticommutation relations
%
\begin{equation}
\{\eta^\dagger_q,c^\dagger_j\}=\{\eta_q,c_j\}=0
\end{equation}
%
and
%
\begin{equation}
\{\eta_q^\dagger,c_j\}=\Phi_q(j)\delta_{k,j},\quad\{\eta_q,c^\dagger_j\}=\Phi^*(j)
\delta_{k,j}
\end{equation}
%
These imply that the expectation value of the two-point function in a generic 
eigenstate of~\eqref{xx_ham} is given as 
%
\begin{equation}
\langle c_i^\dagger c_j\rangle=\sum_{q}\Phi_q^*(i)\Phi_q(j), 
\end{equation}
%
where the sum if over the $q$ single-particle excitations forming the 
eigenstate. 

One can show that the eigenvalus of the matrix $T$ are organized in pairs with 
opposite sign. Given the components $\Phi_1(i)$ of the eigenvector with eigenvalue 
$\epsilon_q$, the components of the eigenvector with $-\epsilon_q$ are given 
as $(-1)^{i+1}\Phi_q(i)$. This also implies that the ground state of~\eqref{xx_ham} 
is in the sector with $M=L/2$ fermions. 

The reduced density matrix for a subsystem $A$ is determined by the correlation 
matrix restricted to $A$ as 
%
\begin{equation}
{\mathcal C}_{ij}\equiv\langle c_i^\dagger c_j\rangle, 
\end{equation}
%
where $i,j\in A$. In particular, given the eigenvalues $\lambda_k$ of ${\mathcal C}$, 
the entanglement entropy $S_A$ is given as 
%
\begin{equation}
S_A=-\sum_k(\lambda_k\log\lambda_k+(1-\lambda_k)\log(1-\lambda_k)).
\end{equation}
%

%%%%%%%%%%%%%%%%%%%%%%%%%%%%%%%%%e
\begin{figure}[t]
\includegraphics*[width=0.99\linewidth]{./draft_figs/rXX_exact}
\caption{
}
\label{fig1}
\end{figure}
%%%%%%%%%%%%%%%%%%%%%%%%%%%%%%%%%%

%########################################################################
\section{The entanglement negativity in DMRG}
\label{neg-dmrg}

Here we show how to calculate the logarithmic negativity in DMRG simulations. 

We consider a generic $1D$ lattice of $L$ sites. We restrict ourselves to the 
situation with open boundary conditions. On each site $i$ of the lattice we 
consider a local Hilbert space $\mathbb{H}_i$ of finite dimension $d$ (e.g., for 
spin-$1/2$ systems one has $d=2$). A generic wavefunction $|\Psi\rangle$ corresponding 
to a pure stae living in $\mathbb{H}^{\otimes L}$ can be written in Matrix-Product-State 
(MPS) language as 
%
\begin{equation}
|\Psi\rangle=\sum\limits_{\sigma_1,\sigma_2,\dots,\sigma_L}A^{\sigma_1}_{\alpha_1}
A^{\sigma_2}_{\alpha_1\alpha_2}\cdots A^{\sigma_L}_{\alpha_L}|\sigma_1,\dots,
\sigma_L\rangle.
\label{mps} 
\end{equation}
%
Here $1\le\sigma_i\le d$ is the so-called physical index that labels the states 
in the local Hilbert space, whereas $1\le\alpha_i\le\chi_i$ are the virtual 
indices, and $\chi_i$ is the bond dimension of the MPS. In~\eqref{mps} we the sum 
over the repeated indices $\alpha_i$ is assumed and $|\sigma_1,\sigma_2,\dots,
\sigma_L\rangle\equiv|\sigma_1\rangle\otimes|\sigma_2\rangle\cdots|\sigma_L\rangle$, 
with $|\sigma_i\rangle$ an element of the base of the Hilbert space at site 
$i$. 

At each site $1<i<L$, for each fixed $\sigma_i$, $A^{\sigma_i}_{\alpha_{i-1}\alpha_i}$ 
is a $\chi_i\times\chi_i$ matrix, while $A^{\sigma_1}_{\alpha_1}$ and 
$A^{\sigma_L}_{\alpha_L}$ are vectors. 

The MPS in~\eqref{mps} can be pictorially represented as in Fig. .... From~\eqref{mps} 
one obtains that $\langle\Psi|$ is given as 
%
\begin{equation}
\langle\Psi|=\langle\sigma_1,\dots,\sigma_L| \sum\limits_{\sigma_1,\sigma_2,
\dots,\sigma_L}\bar A^{\sigma_1}_{\alpha_1}\bar A^{\sigma_2}_{\alpha_1\alpha_2}\cdots 
\bar A^{\sigma_L}_{\alpha_L}
\label{mps_conj} 
\end{equation}
%
where the bar in $\bar A^{\sigma_i}_{\alpha_{i-1}\alpha_i}$ represents the complex 
conjugation. 

The full system density matrix $\rho\equiv|\Psi\rangle\langle\Psi|$ is obtained as
%
\begin{widetext}
%
\begin{equation}
\rho=\sum\limits_{\substack{\sigma_1,\dots,\sigma_L\\\sigma_1',\dots,\sigma_L'}}
(A^{\sigma_1}_{\alpha_1}\bar A^{\sigma_1'}_{\alpha_1'})(A^{\sigma_2}_{\alpha_1,\alpha_2}
\bar A^{\sigma_2'}_{\alpha_1',\alpha_2'})\cdots (A^{\sigma_L}_{\alpha_L}
\bar A^{\sigma_L'}_{\alpha_L'})|\sigma_1\rangle\langle\sigma_1'|\otimes\cdots\otimes
|\sigma_L\rangle\langle\sigma_L'|. 
\end{equation}
%
\end{widetext}
%
The pictorial representation of $\rho$ is shown in Fig. ... . 

We now introduce a tripartition of the system as $S=A_1\cup A_1\cup B$, where $A_1$ and 
$A_2$ are the subintervals of interest and $B$ the remaining part of the system. 
To be specific we consider the case of two intervals of length $\ell_1$ and $\ell_2$ 
at mutual distance $d$ with $A_1=[s+1,\dots,s+\ell_1]$ $A_2=[s+\ell_1+d+1,\dots,s+
\ell_1+d+\ell_2]$ with $A_1$ shifted from the left boundary of the chain by $s$ sites.
The reduced density matrix $\rho_{A1\cup A_2}$ is obtained by tracing over the degrees 
of freedom of $B$ as 
%
%\begin{widetext}
%\begin{multline}
%\rho_{A_1\cup A_2}=\sum\limits_{\substack{\{\sigma^{(1)}_i,\sigma'^{(1)}_{i}\\
%\sigma^{(2)}_j,\sigma'^{(2)}_{j}\}}}
%E^{\alpha,\beta}_{\alpha',\beta'}
%A^{\sigma^{(1)}_1}_{\gamma,\nu_1}\bar A^{\sigma'^{(1)}_1}_{\gamma,\nu'_1}
%\Big[\prod\limits_{n=2}^{\ell_1-1}A^{\sigma^{(1)}_n}_{\nu_{n-1},\nu_n}
%\bar A^{\sigma'^{(1)}_n}_{\nu'_{n-1},\nu'_n}\Big]
%A^{\sigma^{(1)}_{\ell_1}}_{\nu_{\ell_1-1},\alpha}\bar 
%A^{\sigma'^{(1)}_{\ell_1}}_{\nu'_{\ell_1-1},\alpha'}\\
%A^{\sigma^{(2)}_1}_{\beta,\mu_1}\bar A^{\sigma'^{(2)}_1}_{\beta',\mu'_1}
%\Big[\prod\limits_{n=2}^{\ell_2-1}A^{\sigma^{(2)}_n}_{\mu_{n-1},\mu_n}
%\bar A^{\sigma'^{(2)}_n}_{\mu'_{n-1},\mu'_n}\Big]
%A^{\sigma^{(2)}_{\ell_2}}_{\mu_{\ell_2-1},\gamma'}\bar 
%A^{\sigma'^{(2)}_{\ell_2}}_{\mu'_{\ell_2-1},\gamma'}\prod\limits_{i\in A_1\cup A_2}
%|\sigma_i\rangle\langle\sigma'_i|.
%\end{multline}
%\end{widetext}
%
%
\begin{widetext}
\begin{multline}
\label{rho12}
\rho_{A_1\cup A_2}=\sum\limits_{\substack{\{\sigma_i,\sigma'_{i}:\, i\in A_1\cup A_2\}}}
\delta_{\nu_1,\nu_1'}\underbrace{\left(\prod\limits_{n\in A_1}A^{\sigma_n}_{\nu_{n},\nu_{n+1}}\right)}_{F_1}
\left(\prod_{n\in A_1}\bar A^{\sigma'_n}_{\nu'_{n},\nu'_{n+1}}\right)
[E_{12}]^{\nu_{\ell_1+1},\nu_{\ell_1+d+1}}_{\nu'_{\ell_1+1},\nu'_{\ell_1+d+1}}\\
\underbrace{\left(\prod\limits_{m\in A_2}A^{\sigma_m}_{\nu_{m},\nu_{m+1}}\right)}_{F_2}
\left(\prod_{m\in A_2}\bar A^{\sigma'_m}_{\nu'_{m},\nu'_{m+1}}\right)
\delta_{\nu_{\ell_1+\ell_2+d+1},
\nu'_{\ell_1+\ell_2+d+1}}\prod\limits_{i\in A_1\cup A_2}
|\sigma_i\rangle\langle\sigma'_i|.
\end{multline}
\end{widetext}
%
Here the index $n\in[1,\dots,\ell_1]$ ($m\in[d+\ell_1+1,\dots,d+\ell_1+\ell_2]$) counts 
only the spins in the interval $A_1$ ($A_2$). Notice that the two Kronecker delta 
in~\eqref{rho12} are because the virtual indices $\nu_1,\nu'_1$ and 
$\nu_{\ell_1+\ell_2+d+1},\nu'_{\ell_1+\ell_2+d+1}$ are contracted. They arise from 
contracting the matrices $A$ lying outside the interval of interest. Moreover, repeated 
indices are summed over, and the term in the first square bracket in~\eqref{rho12} 
depends on $\nu_{\ell_1+1}$ and $\nu'_{\ell_1+1}$. Finally, the tensor $F_1$ depends on 
the physical indices $\sigma_1,\dots,\sigma_{\ell_1}$ and on $\nu_1,\nu_{\ell_1+1}$.
In~\eqref{rho12} we defined the ``transfer matrix'' between the two intervals $E$ as 
%
\begin{equation}
\label{E}
[E_{12}]^{\nu_{\ell_1+1},\nu_{\ell_1+d+1}}_{\nu'_{\ell_1+1},\nu'_{\ell_1+d+1}}\equiv 
\prod\limits_{j\in B_1}A^{\sigma_j}_{\nu_j,\nu_{j+1}}
\bar A^{\sigma_j}_{\nu'_j,\nu'_{j+1}}. 
\end{equation}
%
Notice that all the spin indices are contracted in~\eqref{E}. Also, the computational 
cost for constructing $E$ is ${\mathcal O}(\chi^6_{max})$, with $\chi_{max}$ the 
largest bond dimension in the chain. 
It is also convenient for later to define the analogous tensors $E_1$ and $E_2$ for 
the intervals $A_1$ and $A_2$ as 
%
\begin{align}
& [E_1]^{\nu_1,\nu_{\ell_1+1}}_{\nu'_1,\nu'_{\ell_1+1}}\equiv
\prod\limits_{j\in A_1}A^{\sigma_j}_{\nu_j,\nu_{j+1}}
\bar A^{\sigma_j}_{\nu'_j,\nu'_{j+1}}\\
& [E_2]^{\nu_{\ell_1+d+1},\nu_{\ell_1+\ell_2+d+1}}_{\nu'_{\ell_1+d+1},
\nu'_{\ell_1+\ell_2+d+1}}\equiv
\prod\limits_{j\in A_2}A^{\sigma_j}_{\nu_j,\nu_{j+1}}
\bar A^{\sigma_j}_{\nu'_j,\nu'_{j+1}}.
\end{align}
%
It is convenient to recast~\eqref{rho12} as 
%
\begin{widetext}
\begin{multline}
\rho_{A_1\cup A_2}=\sum_{\sigma_i:\, i\in A_1}
[F_1]^{\sigma_1,\dots,\sigma_{\ell_1}}_{\nu_1,\nu_{\ell_1+1}}\prod_{j\in A_1}|\sigma_j\rangle
\langle\sigma'_j|\sum_{\sigma'_i:\, i\in A_1}
[\bar F_1]^{\sigma'_1,\dots,\sigma'_{\ell_1}}_{\nu'_1,\nu'_{\ell_1+1}}\,
[E_{12}]^{\nu_{\ell_1+1},\nu_{\ell_1+d+1}}_{\nu'_{\ell_1+1},\nu'_{\ell_1+d+1}}
\delta_{\nu_{\ell_1+\ell_2+d+1},
\nu'_{\ell_1+\ell_2+d+1}}\delta_{\nu_1,\nu'_1}
\\\otimes\sum_{\sigma_i:\, i\in A_2}
[F_2]^{\sigma_{\ell_1+d+1},\dots,\sigma_{\ell_1+\ell_2+d}}_{\nu_{\ell_1+d+1},
\nu_{\ell_1+\ell_2+d+1}}\prod_{j\in A_2}|\sigma_j\rangle\langle\sigma'_j|
\sum_{\sigma'_i:\, i\in A_2}
[\bar F_2]^{\sigma'_{\ell_1+d+1},\dots,\sigma'_{\ell_1+\ell_2+d}}_{\nu'_{\ell_1+d+1},
\nu'_{\ell_1+\ell_2+d+1}}. 
\end{multline}
\end{widetext}
%
This can be rewritten in a more convenient form as 
%
\begin{widetext}
\begin{multline}
\label{rho12-s}
\rho_{A_1\cup A_2}=\sum_{\sigma_i:\, i\in A_1}
[F_1]^{\{\sigma_i\}}_{\nu_{1L},\nu_{1R}}\prod_{j\in A_1}|\sigma_j\rangle
\langle\sigma'_j|\sum_{\sigma'_i:\, i\in A_1}
[\bar F_1]^{\{\sigma'_i\}}_{\nu'_{1L},\nu'_{1R}}\,
[E_{12}]^{\nu_{1R},\nu_{2L}}_{\nu'_{1R},\nu'_{2L}}
\delta_{\nu_{2R},
\nu'_{2R}}\delta_{\nu_{1L},\nu'_{1L}}
\\\otimes\sum_{\sigma_i:\, i\in A_2}
[F_2]^{\{\sigma_{i}\}}_{\nu_{2L},
\nu_{2R}}\prod_{j\in A_2}|\sigma_j\rangle\langle\sigma'_j|
\sum_{\sigma'_i:\, i\in A_2}
[\bar F_2]^{\{\sigma'_i\}}_{\nu'_{2L},
\nu'_{2R}},
\end{multline}
\end{widetext}
%
where we defined the indices $\nu_{1L}\equiv\nu_1,\nu_{1R}\equiv\nu_{\ell_1+1}$ 
and $\nu_{2L}\equiv\nu_{\ell_1+1},\nu_{2R}\equiv\nu_{\ell_1+\ell_2+d+1}$. 

After introducing the states
%
\begin{align}
|w^{(1)}_{\nu_{1L},\nu_{1R}}\rangle\equiv\sum\limits_{\sigma_i:\,i\in A_1}
[F_1]^{\{\sigma_i\}}_{\nu_{1L},\nu_{1R}}\prod\limits_{j\in A_1}|\sigma_j\rangle,\\
|w^{(2)}_{\nu_{2L},\nu_{2R}}\rangle\equiv\sum\limits_{\sigma_i:\,i\in A_2}
[F_2]^{\{\sigma_i\}}_{\nu_{2L},\nu_{2R}}\prod\limits_{j\in A_2}|\sigma_j\rangle.
\end{align}
%
Eq.~\eqref{rho12-s} can be rewritten in the compact form  
%
\begin{widetext}
\begin{equation}
\rho_{A_1\cup A_2}=\delta_{\nu_{1L},\nu'_{1L}}\delta_{\nu_{2R},\nu'_{2R}}
[E_{12}]^{\nu_{1R},\nu_{2L}}_{\nu'_{1R},\nu'_{2L}}|w^{(1)}_{\nu_{1L},\nu_{1R}}
\rangle\langle w^{(1)}_{\nu'_{1L},\nu'_{1R}}|\otimes |w^{(2)}_{\nu_{2L},\nu_{2R}}
\rangle\langle w^{(2)}_{\nu'_{2L},\nu'_{2R}}|.
\end{equation}
\end{widetext}
%
One should now observe that, although the vectors $|w^{(1)}_{\nu_{1L},\nu_{1R}}
\rangle$ provide a basis for the interval $A_1$ (similarly for $A_2$) they are 
not orthonormal (not even orthogonal). Clearly, one has that 
%
\begin{align}
\langle w^{(1)}_{\nu_{1L},\nu_{1R}}|w^{(1)}_{\nu'_{1L},\nu'_{1R}}\rangle=
[E_1]^{\nu_{1L},\nu_{1R}}_{\nu'_{1L},\nu'_{1R}} \\
\langle w^{(2)}_{\nu_{2L},\nu_{2R}}|w^{(2)}_{\nu'_{2L},\nu'_{2R}}\rangle=
[E_2]^{\nu_{2L},\nu_{2R}}_{\nu'_{2L},\nu'_{2R}}. 
\end{align}
%  
A notable exception when the basis given by $|w^{(1)}_{\nu_{1L},\nu_{1R}}\rangle$ 
and $|w^{(2)}_{\nu_{2L},\nu_{2R}}\rangle$ is orthogonal is when the two intervals 
$A_1$ and $A_2$ are attached at the left and right edges of the chain, respectively. 
In this situation the reduced density matrix is given as 
%
\begin{equation}
[\rho_{A_1\cup A_2}]^{(\nu_{1R},\nu_{2L})}_{(\nu'_{1R},\nu'_{2L})}=
[E_{12}]^{\nu_{1R},\nu_{2L}}_{\nu'_{1R},\nu'_{2L}}
\end{equation}
%
and it is a $\chi^2_{max}\times\chi^2_{max}$ matrix. 
In the more general case, one way to promote the basis to an orthonormal one is 
by performing a singular value decomposition (SVD) of the matrices $E_1$ and $E_2$. 
Given the SVD of $E_1$ as $E_1=UDU^\dagger$ with $U$ orthogonal and $D$ diagonal, 
then the vectors $|v^{(1)}_m\rangle$ 
%
\begin{equation}
|v^{(1)}_{m}\rangle\equiv [N_1]^m_{\nu_{1L},\nu_{1R}}|w^{(1)}_{\nu_{1L},\nu_{1R}}\rangle
\end{equation}
%
with $N_1\equiv D^{-\frac{1}{2}}U$ are orthonormal. This leads to 
%
\begin{widetext}
\begin{equation}
\rho_{A_1\cup A_2}=
\end{equation}
\end{widetext}
%




%##########################################################################
\begin{thebibliography}{99}

\bibitem{caux-2009}

F.~Igl\'oi, R.~Juh\'asz, and H.~Rieger, Phys.\ Rev.\ B {\bf 61}, 11552 (2000).

\bibitem{pippan-2010}
P.~Pippan, S.~R.~White, and H.~G.~Evertz, Phys.\ Rev.\ B {\bf 81}, 081103(R) 
(2010). 

\bibitem{laflorencie-2005}
N.~Laflorencie, Phys.\ Rev.\ B {\bf 72}, 140408(R) (2005). 

\end{thebibliography}

\end{document}



