\documentclass[twocolumn,superscriptaddress,prb,10pt]{revtex4-1}
%\usepackage{verbatim}
\usepackage{amsmath,amssymb}
%\usepackage{unicode-math}
\usepackage{graphicx}
\usepackage{color}
\usepackage[colorlinks,bookmarks=false,citecolor=blue,linkcolor=red,urlcolor=blue]{hyperref}
\usepackage{times}





%\usepackage[dvips]{graphics}



%%%%%%%%%%%%%   useful shortcuts %%%%%%%%%%%%%%%%%%%%%%%%%%%%%%%%%

\def \h{\hbar}   %  \h won't be used for any greek letter
\def \refe{\eqref}
\def \trm{\textrm}
\def \f{\frac}
\def \hf{\tfrac{1}{2}}    \def \HF{\dfrac{1}{2}}
\def \u{\uparrow}
\def \d{\downarrow}

\def \ord{\mathcal{O}}
\newcommand{\ra}{\rightarrow}   \newcommand{\lra}{\longrightarrow}  

\def\lba{\left(}    \def\rba{\right)}
\def\lbc{\left[}    \def\rbc{\right]}
\def\lbb{\left\{}    \def\rbb{\right\}}

\def\tr{\textrm{Tr}}
\def\refe{\eqref}

\newcommand{\bra}[1]{\langle\left.{#1}\right|}
\newcommand{\ket}[1]{\left|{#1}\right.\rangle}
\newcommand{\xpct}[1]{\langle{#1}\rangle}    % expectatn value

%\DeclareMathOperator{\tr}{tr}

%%%%%%%%%%%%%%%%%%%%%%%%%%%%%%%%%%%%%%%%%%%%%%%%%%%%%%%

\newcommand{\vp}{{\bf p}}  % usual vector quantities
\newcommand{\vq}{{\bf q}}  % double bracketing not required with \vec
\newcommand{\vk}{{\bf k}}  % but required with \bf

\renewcommand{\vr}{{\bf r}} 
\newcommand{\vx}{{\bf x}}

\newcommand{\hc}{\hat{c}}  \newcommand{\hcd}{\hat{c}^\dag} 
\newcommand{\hd}{\hat{d}}  \newcommand{\hdd}{\hat{d}^\dag} 






\begin{document}

\title{The logarithmic negativity in random spin chains} 

%\author{Vincenzo Alba}
%\affiliation{International School for Advanced Studies (SISSA),
%Via Bonomea 265, 34136, Trieste, Italy, 
%INFN, Sezione di Trieste}

\author{Authors}

\date{\today}




\begin{abstract} 


\end{abstract}

% \pacs{73.43.Cd, 71.10.Pm  {\tt check!}}

\maketitle


%########################################################################
\section{Introduction}

%%%%%%%%%%%%%%%%%%%%%%%%%%%%%%%%%e
%\begin{figure*}[t]
%\includegraphics*[width=0.99\linewidth]{./draft_figs/}
%\caption{
%}
%\label{fig1}
%\end{figure*}
%%%%%%%%%%%%%%%%%%%%%%%%%%%%%%%%%%


%########################################################################
\section{The disordered $XX$ chain}
\label{dis-XX}

The disordered $XX$ chain is defined by the Hamiltonian 
%
\begin{equation}
{\mathcal H}_{XX}=\sum\limits_{i=1}^{L-1}J_i(S^x_iS^x_{i+1}+S^y_i
S^y_{i+1})+h\sum\limits_{i=1}^{L}S_i^z, 
\label{xx_ham}
\end{equation}
%
with $S^{x,y,z}_i\equiv\sigma_i^{x,y,z}/2$, $\sigma_i^\alpha$ 
being the Pauli matrices acting on site $i$. For periodic 
boundary conditions one has an extra term in Eq.~\eqref{xx_ham} 
connecting site $L$ with site $1$. Hereafter we fix $h=0$ and 
choose $J_i$ uniformily distributed in $[0,1]$. After the Jordan-Wigner 
transformation 
%
\begin{equation}
c_i=\Big(\prod\limits_{m=1}^{i-1}\sigma^z_m\Big)
\frac{\sigma_i^x-i\sigma_i^y}{2},
\label{j-wigner}
\end{equation}
%
\eqref{xx_ham} is recast in the free-fermionic form 
%
\begin{equation}
{\mathcal H}_{XX}=\frac{1}{2}\sum\limits_{i=1}^{L-1}J_i(c^\dagger_i 
c_{i+1}+c^\dagger_{i+1}c_i)+\frac{h}{2}\sum\limits_{i=1}^{L-1}
c^\dagger_i c_i,
\label{xx_fer}
\end{equation}
%
with $c_i$ spinless fermionic operators satisfying the canonical 
anticommutation relations $\{c_m,c^\dagger_n\}=\delta_{m,n}$. 
The mapping between Eq.~\eqref{xx_ham} and Eq.~\eqref{xx_fer} 
is exact apart from boundary terms (that we neglect here) giving a 
vanishing contribution (as $1/L$) to physical quantities in the  large 
chain limit. 

By imposing that the single-particle eigenstates $|\Psi_q\rangle$ (with $q$ 
an integer labelling the different eigenstates) of~\eqref{xx_fer} are of 
the form 
%
\begin{equation}
\eta_q^\dagger|0\rangle\equiv|\Psi_q\rangle=\sum_i\Phi_q(i)c_i^\dagger|0\rangle,
\end{equation}
%
with $|0\rangle$ the vacuum and $\Phi_q(i)$ the eigenstate amplitudes. 
The Schr\"odinger equation gives the equation for $\Phi_q(i)$ as 
%
\begin{equation}
\label{xx-eig}
(J_i\Phi_q(i+1)+J_{i-1}\Phi_q(i-1))/2=\epsilon_q\Phi_q(i),\quad i=1,\dots,L, 
\end{equation}
%
and $J_L=0$. Eq.~\eqref{xx-eig} corresponds to finding the eigenvectors 
of the banded matrix $T=(J_j\delta_{i,j+1}+J_{j-1}\delta_{i,j-1})/2$. The eigenvalues 
corresponds to the single-particle eigenergies $\epsilon_q$. 

The ground state $|GS\rangle$ of~\eqref{xx_ham} is obtained by filling all the 
negative modes $\epsilon_q$ as 
%
\begin{equation}
|GS\rangle=\eta^\dagger_{q_M}\eta^\dagger_{q_{M-1}}\cdots\eta_{q_1}^\dagger|0\rangle.
\end{equation}
%
One has the anticommutation relations
%
\begin{equation}
\{\eta^\dagger_q,c^\dagger_j\}=\{\eta_q,c_j\}=0
\end{equation}
%
and
%
\begin{equation}
\{\eta_q^\dagger,c_j\}=\Phi_q(j)\delta_{k,j},\quad\{\eta_q,c^\dagger_j\}=\Phi^*(j)
\delta_{k,j}
\end{equation}
%
These imply that the expectation value of the two-point function in a generic 
eigenstate of~\eqref{xx_ham} is given as 
%
\begin{equation}
\langle c_i^\dagger c_j\rangle=\sum_{q}\Phi_q^*(i)\Phi_q(j), 
\end{equation}
%
where the sum if over the $q$ single-particle excitations forming the 
eigenstate. 

One can show that the eigenvalus of the matrix $T$ are organized in pairs with 
opposite sign. Given the components $\Phi_1(i)$ of the eigenvector with eigenvalue 
$\epsilon_q$, the components of the eigenvector with $-\epsilon_q$ are given 
as $(-1)^{i+1}\Phi_q(i)$. This also implies that the ground state of~\eqref{xx_ham} 
is in the sector with $M=L/2$ fermions. 






%##########################################################################
\begin{thebibliography}{99}

\bibitem{caux-2009}

F.~Igl\'oi, R.~Juh\'asz, and H.~Rieger, Phys.\ Rev.\ B {\bf 61}, 11552 (2000).


\end{thebibliography}

\end{document}



